
\chapter{Интеркалирование графена на Ni(111)}\label{ch:ch1}
Здесь должны быть описаны эксперименты по графену на никеле.

В экспериментах я участия не принимала, но зато мы с Гошей написали статью по экспериментам и расчетам - \cite{grebenyuk2019840}, которую я в этом разделе и пересказываю.
\section{Экспериментальная часть}\label{sec:ch1/sec1}


Эксперименты проводились на месте в условиях сверхвысокого вакуума. В качестве образцов использовались монокристаллы W (110). Их поверхность очищалась отжигом в кислороде с последующим быстрым нагревом до температуры ~ 2000°С в сверхвысоком вакууме. Затем на поверхности W (110)  выращивались эпитаксиальные пленки Ni (111) толщиной от 8 до 12 нм. Высококачественный графен формировался на этих подложках путем химического осаждения из паровой фазы (CVD). Для этого образцы были выдержаны в атмосфере $C_3H_6$ при давлении 2×10$^{-6}$ мбар в течение 10 минут при 600°C как описано в литературе [4, 13, 14]. Интеркалирование графена кобальтом и кремнием осуществлялось путем осаждения пленок Co и Si толщиной от 0,1 до 3,5 нм при комнатной температуре и последующего отжига образцов при различных температурах в диапазоне от 300 до 550° C. Толщина осажденных пленок контролировалась с помощью кварцевых микровесов и фотоэлектронной спектроскопии. 
Элементный состав и химическое состояние поверхностей образцов, а также их атомная структура контролировались методами фотоэлектронной спектроскопии с высоким энергетическим разрешением и дифракции медленных электронов. Энергия фотонов варьировалась в пределах 80–600 эВ. Фотоэлектронные спектры снимали на электронно-энергетическом анализаторе PHOIBOS 150. Полное энергетическое разрешение спектрометра составляло около 100 мэВ. Все спектры были получены при комнатной температуре и базовом давлении лучше, чем 2 × 10-10 мбар. 

Изменения интенсивности линий Ni 3$p$ и C 1$s$ в ходе эксперимента показаны на Рисунке \ref{img:ni1}. Рост однослойного графена на поверхности никеля приводит к затуханию линии Ni 3p и появлению пика C 1s. , Осаждение при комнатной температуре 1 монослоя (ML) кобальта на графене ослабляет интенсивность линий C 1s и Ni 3p (рис. 1) и приводит к появлению линии Co 3p. Последующий отжиг образца при 400C в течение 10 минут восстанавливает интенсивность линии C 1s и снижает интенсивность Co 3p, в то время как интенсивность линии Ni 3p остается постоянной. Такая динамика спектров на уровне ядра свидетельствует об интеркаляции кобальта под графеном [4, 5, 12]. Механизм интеркалирования графена с атомами кобальта следующий: после повышения температуры атомы, осажденные поверх графена, начинают мигрировать к границам доменов и проникают через графен, образуя слой кобальта между графеном и никелевой пленкой.
Увеличение количества нанесенного кобальта до 5 мл с последующим отжигом при 400 ° С не меняет интенсивность линии C 1s (рис. 1), что указывает на дальнейшую интеркаляцию графена с кобальтом. Данные, представленные на рис. 1, показывают, что такой метод интеркаляции позволяет формировать достаточно толстые (до 17 мл) пленки кобальта под графеном.

Схема LEED, полученная для системы Gr / Ni (111), показана на рис. 2а. Это демонстрирует хорошо упорядоченную p (1 × 1) структуру графена на Ni (111), вызванную небольшим несоответствием (1,2\%) их решеток. Высокая яркость дифракционных пятен подтверждает высокое качество выращенного графена. На рис. 2, б показаны образцы LEED, наблюдаемые после интеркаляции графена с кобальтом. Даже образование Co-пленки толщиной 17 ML не меняет дифракционную картину. Незначительное уменьшение интенсивности дифракционных пятен можно объяснить появлением дефектов в графеновом слое. Сохранение дифракционной картины означает, что интеркалированная пленка кобальта имеет такую же ГЦК-кристаллическую структуру, что и пленка Ni (111). Полученные данные согласуются с результатами работы. [12], где псевдоморфный рост слоев ГЦК-железа наблюдался в системе Gr / Fe / Ni (111) в диапазоне покрытия до 14 мл Fe. 

На рис. 3 показаны изменения формы линии C 1s на разных этапах эксперимента. Спектр системы Gr / Ni (111) согласуется с ранее опубликованными данными [4, 5, 7, 15]. Энергия связи электронов C 1s равна 284,9 эВ, что смещено на 0,5 эВ в сторону более высоких значений по сравнению с линией автономного графена. Это объясняется тем, что графен, будучи помещенным на металлическую подложку, сильно связывается с подложкой из-за гибридизации углеродных p-электронов с металлическими d-электронами. Интеркаляция графена с кобальтом не меняет энергетическое положение пика C 1s, но меняет его ширину. Величина уширения увеличивается с количеством интеркалированного кобальта. В работе [16] было обнаружено, что появление дефектов в графене меняет расположение соседних атомов, что приводит к сдвигу энергий связи электронов C 1s. Этот сдвиг приводит к расширению линии C 1s. Таким образом, мы предполагаем, что в нашем случае интеркаляция кобальта немного увеличивает количество дефектов в графеновом слое. 
Влияние интеркаляции кобальта на спектры валентной зоны системы Gr / Co / Ni (111) показано на рис. 4. Начальный спектр Gr / Ni (111) показан в нижней части рисунка. Вблизи уровня Ферми он имеет форму, типичную для 3d-состояний Ni, а при 9,7 эВ он демонстрирует максимальное соответствие π-состояниям графена [17]. Осаждение 1 мл Со приводит к ослаблению этого свойства, а последующий отжиг образца приводит к восстановлению этого свойства. Дальнейшая интеркаляция кобальта не влияет на энергетическое положение или амплитуду пика графена, но меняет форму максимума вблизи уровня Ферми, который становится похожим на пик пленки кобальта. Следует отметить, что фиксированные положения линий C 1s и π-состояний указывают на постоянную сильную связь между графеновым слоем и подложкой во время интеркаляции кобальта. 
3.2. Интеркаляция графена / кобальта / никеля с кремнием (Intercalation of graphene/cobalt/nickel with silicon)
Спектр Si 2p, полученный после осаждения при комнатной температуре 0,6 мл Si на кобальт-интеркалированный образец, представляет собой довольно широкую и бесформенную кривую с максимумом при энергии связи 99,3 эВ, что характерно для островков аморфного Si на вершине графен [12]. Отжиг образца при 400 ° С приводит к значительному уменьшению интенсивности этой особенности и появлению хорошо разрешенных линий (рис. 5). Учитывая, что интенсивность линии C 1s (см. Правую часть рис. 1) остается постоянной, мы приходим к выводу, что состояние после отжига может быть описано как полная интеркаляция атомов Si под графеном. 

Как показано в верхней части рисунка 5, спектр Si 2p после интеркаляции 0,6 мл Si может быть разложен на три спин-орбитальных дублета (моды B, C и S). Это указывает на то, что в анализируемом слое имеется три типа атомов кремния. Эти режимы можно интерпретировать, используя доступные опубликованные данные. Спектры Si 2p, измеренные после интеркаляции графена, образованного на монокристаллическом кобальте с атомами кремния, представлены в [7, 10]. Эти спектры состоят из двух мод с энергиями 98,9 эВ и 99,3 эВ. В работе [10] было показано, что эти моды соответствуют атомам кремния, адсорбированным на кобальте, и твердому раствору кремния в кобальте соответственно. В нашем случае энергетические положения наиболее интенсивных B- и S-мод (99,31 и 98,91 эВ, рис. 5) такие же, как в описанных работах. Поэтому можно предположить, что B- и S-моды соответствуют этим двум фазы. Что касается третьего компонента C (99,1 эВ), который не наблюдался в [7, 10], мы полагаем, что он возникает из-за более активной диффузии кремния в интеркалированный слой кобальта по сравнению с монокристаллическим кобальтом, который может привести к образованию силицида кобальта. Поскольку компонент с такой же энергией связи ранее наблюдался при образовании силицида Co2Si на кремнии [18], мы полагаем, что третий компонент C соответствует силициду Co2Si.

Информация об атомной структуре поверхностного силицида кобальта может быть получена по схеме LEED, полученной после интеркалирования 0,6 мл Si (рис. 2в). Рисунок демонстрирует не только p (1 × 1) структуру покрытой графеном пленки кобальта, но также и дополнительные пятна сверхструктуры (√3 × √3) R30 °, вызванной упорядоченным поверхностным силицидом. Следует отметить, что такая же картина LEED наблюдалась в нашей предыдущей работе по системе Gr / Si / Fe (111), когда под действием графена формировался поверхностный силицид Fe3Si [12]. Совпадение дифракционных картин может указывать на идентичность элементного состава поверхностных фаз. Поэтому полагаем, что образовавшаяся поверхность силицида кобальта имеет состав Co3Si.

Дальнейшая интеркаляция кремния не приводит к появлению новых мод спектра Si 2p, а только увеличивает его общую интенсивность. В этом случае абсолютная интенсивность S-компонента остается неизменной, а интенсивность B- и C-компонентов значительно увеличивается (рис. 5). Это означает, что толщина поверхностного слоя Co3Si постоянна, а слои Co-Si и Co2Si становятся толще. Таким образом, увеличение количества интеркалированного кремния позволяет изменять толщину синтезированных пленок силицидов.

Данные, представленные на рис. 5, показывают, что увеличение температуры отжига до 580C приводит к значительному ослаблению интенсивностей компонентов B и C, что означает, что твердый раствор Co-Si и силицид Co2Si частично растворяются в глубина образца. Таким образом, изменение температуры отжига также позволяет варьировать состав пленок силицида кобальта.

Спектры C 1s, измеренные после интеркаляции различных количеств кремния под графеном, показаны в верхней части рисунка 3. После интеркаляции первой части Si линия C 1s сдвигается к нижней энергии связи, равной 284,4 эВ. Последующее увеличение дозы интеркаляции приводит к дальнейшему небольшому сдвигу линии. Наблюдаемое изменение энергии связи C 1s указывает на модификацию химических состояний атомов углерода, которые становятся похожими на состояния в квази- автономном графене [19].

Влияние интеркаляции графена с Si на спектры валентной зоны проиллюстрировано в верхней части рис. 4. Как отмечалось выше, до осаждения Si максимальное соответствие π-состояниям графена наблюдалось при 9,7 эВ. После интеркаляции образца 0,6 мл Si он сдвигался до 7,8 эВ, что близко к его положению в отдельно стоящем графене [17]. Таким образом, сдвиг этой особенности в спектрах валентной зоны согласуется с изменением энергии связи линии C 1s, описанным выше. В целом, эти результаты ясно демонстрируют модификацию электронной структуры слоя графенового покрытия из-за его интеркаляции с Si.





4. Вывод (Conclusions)
Предложен новый метод синтеза покрытых графеном силицидов кобальта, который включает формирование графена на поверхности Ni (111) с последующей интеркалированием графена кобальтом и кремнием, что происходит при повышенной температуре. Основными преимуществами этой процедуры являются более высокое качество графена по сравнению с графеном, выращенным на кобальте, и возможность варьирования состава синтезированных силицидов кобальта. Выявлены особенности процессов интеркаляции в системах Gr / Co / Ni (111) и Gr / Si / Co (111). Интеркаляция графена с кобальтом происходит в широком диапазоне покрытий до 17 мл при температуре 400 ° C. Каппа-графен сильно взаимодействует с атомами кобальта, что стабилизирует ГЦК-структуру пленки Co (111). Последующая интеркаляция Gr / Co / Ni (111) с Si приводит к образованию под графеном трех разных фаз. Предполагается, что эти фазы представляют собой твердый раствор Co-Si и силицид Co2Si, покрытый поверхностной фазой Co3Si структуры (√3 × √3) R30 °. Эта поверхностная фаза представляет собой упорядоченную систему атомов Si, сильно связанных с верхним слоем кобальта. Атомы кремния занимают ГЦК-позиции относительно атомов Co на расстоянии 1,61 Å от плоскости Co (111). Покрывающий графеновый слой слабо связан с Co3Si, сохраняя тем самым исключительные свойства графена, готовые к использованию. Синтезированные силициды кобальта защищены графеном от кислорода, что позволяет использовать эти материалы для применения в наноэлектронике.

\section{Теоретическая часть}\label{sec:ch1/sec1}
А здесь расчеты.

Есть полный набор: железо и кобальт под графеном и их силициды.

3.3. Электронная структура системы графен / кремний / кобальт (Electronic structure of the graphene/silicon/cobalt system)
Чтобы выяснить природу изменения химических состояний атомов углерода, мы провели серию модельных расчетов, относящихся к ранней стадии интеркаляции графена с кремнием. Расчеты ab initio электронной структуры Gr / Si / Co (111) проводились с использованием теории функционала плотности (DFT) и метода псевдопотенциала, реализованного в пакете Quantum Espresso 5.4.0 [20]. Псевдопотенциалы PBE (Perdew-Burke-Ernzerhof) и обобщенное градиентное приближение (GGA) были взяты для обменного функционала энергии корреляции [20]. Для расширения волновых функций валентных электронов использовались плоские волны со значениями энергии в диапазоне до 200 Ry. Чтобы получить самосогласованные расчеты для системы Gr / Si / Co (111), зона Бриллюэна была разделена на 8 × 8 × 1 k-точек методом Монкхорста-Пак. Структурная оптимизация была сделана для каждой атомной конфигурации.

При моделировании надстроек мы учитывали следующие требования. Во-первых, для удовлетворения стехиометрии фазы Co3Si поверхностная плотность атомов Si должна быть в три раза меньше плотности слоя Co (111). Во-вторых, чтобы соответствовать структуре (√3 × √3) R30 °, квадрат суперячейки должен быть в три раза больше, чем площадь суперячейки Gr / Со (111) [21]. Основные конфигурации системы, которые удовлетворяют эти условия показаны на рис. 6. Они соответствуют положениям ГЦК, мостика и вершины атомов Si на поверхности Co (111). Кроме того, эта интеркаляция Si не меняет взаимное положение атомов углерода относительно подложки. Таким образом, они остаются в положениях top-fcc, что характерно для системы Gr / Co (111) [21]. Следовательно, суперячейка состоит из шести атомов C, одного атома Si и восемнадцати атомов Co (шесть слоев с тремя атомами на слой).

Результаты модельных расчетов, выполненных для различных структурных моделей, показаны в Таблице 1. Они демонстрируют, что первая структурная модель является наиболее энергетически выгодной для системы Gr / Si / Co (111). Атомы Si расположены в ГЦК-положениях относительно атомов Co. Расстояние между плоскостью Co (111) и атомами кремния равно 1,61 Å. Следует отметить, что такая структурная модель была также наиболее выгодной для системы Gr / Si / Fe (111) [22], и тот же результат был получен в недавней работе [10].

Важной особенностью оптимальной модели является большое расстояние между атомами углерода и кремния. Графен расположен на расстоянии 2,73 Å от слоя атома кремния. Это значение заметно выше, чем расстояние между графеном и подложкой (2,07 Å) для системы Gr / Co (111) [21]. Такое резкое увеличение межслоевого расстояния приводит к значительному уменьшению перекрытия волновых функций атомов C и атомов подложки, что объясняет ослабление связи между графеном и подложкой, наблюдаемое после интеркаляции Gr / Co. (111) система с кремнием.

Рассчитанная структура электронных зон системы Gr / Si / Co (111) вблизи точки K показана на рис. 7. Для сравнения аналогичные данные для системы Gr / Co (111) также показаны на рис. 7a. Вклады в зонную структуру pz-состояний углерода, которые представляют наибольший интерес, выделены красным. Из рис. 7а видно, что для системы Gr / Co (111) эти вклады вблизи точки K напоминают искаженный конус Дирака, вершина которого существенно ниже уровня Ферми. Кроме того, дисперсионные кривые для большинства и неосновных состояний существенно различаются. Это означает, что интеркаляция графена с кобальтом приводит к возникновению конечной спиновой поляризации углеродных электронных состояний. Интеркаляция системы Gr / Co (111) с кремнием кардинально меняет электронные состояния атомов углерода. Сравнение данных, представленных на рис. 7а и 7б, демонстрирует ослабление гибридизации pz-состояний углерода и 3d-состояний кобальта. Это приводит к исчезновению спиновой поляризации углеродных состояний. Для обеих проекций спина интеркаляция графена с кремнием приводит к существенному смещению конуса Дирака в уровень Ферми. Таким образом, электронная структура графена в значительной степени восстанавливается и становится похожей на структуру квази-автономного графена.