\chapter{Метод функционала плотности} \label{chapt4}

\section{Теоретические основы}
	Одним из самых распространенных инструментов для теоретического анализа систем графен/подложка является метод функционала плотности (density functional theory). Данный подход позволяет моделировать кристаллическое строение исследуемой системы, а также получать информацию о её электронном строении.
	
	Метод функционала плотности лежит в основе современных расчетов электронных свойств твердых тел лежит  Его появление дало серьезный толчок для квантовой химии и вычислительной физики сложных конденсированных систем.
	
	Как известно, квантовая механика дает способ расчета многоэлектронных систем. Однако решение уравнения Шредингера возможно лишь в самых простых случаях. Суть проблемы в том, что волновая функция зависит от 3N переменных, где N — число электронов. Поэтому использование многоэлектронной волновой функции для описания твердых тел оказывается невозможным. 

Идея метода функционала плотности состоит в том, чтобы заменить волновую функцию  электронной плотностью, зависящей лишь от трех пространственных переменных. Возможность такой замены была показана в работе Кона и Хоэнберга в 1964 году \cite{25}. В этой работе были доказаны две теоремы, устанавливающие соответствие между волновой функцией, электронной плотностью и внешним полем. 

Теорема I. Для любой системы взаимодействующих электронов, находящихся во внешнем потенциале  , потенциал  определяется однозначно (с точностью до несущественной константы) электронной плотностью основного состояния  .

	Теорема II. Существует универсальный функционал  электронной плотности, справедливый для любого внешнего потенциала  . Для некоторого вполне определенного внешнего потенциала   экстремум (минимум) полной энергии   достигается для электронной плотности основного состояния n(r).
	
	Эти теоремы имеют важное значение, но не дают практических методов для вычисления наблюдаемых величин. Вид функционала Кона-Хоэнберга может быть установлен только для невзаимодействующего электронного газа или для системы частиц, описываемых в приближении Томаса-Ферми. Практическое значение метод приобрел только после того, как Кон и Шэм предложили подход для вычисления функционала плотности \cite{26}.
	
	Идея состояла в том, чтобы заменить истинный функционал вспомогательным функционалом системы свободных частиц. Вспомогательный гамильтониан выбирается так, чтобы он имел обычную кинетическую энергию и локальный потенциал, ответственный за кулоновское взаимодействие, корреляцию и обмен. 
	
	Уравнение Кона-Шэма выглядит следующим образом:
 	\begin{equation}
  \label{eq:equation1}
 -\frac{1}{2}\triangledown^2\psi_i(r)+v_{KS}(r)\psi_i(r)=\epsilon_i\psi_i(r)
\end{equation}

Оно имеет вид одночастичного уравнения Шрёдингера для частицы, движущейся в самосогласованном потенциале Кона-Шэма:
		 	\begin{equation}
  \label{eq:equation2}
 v_{KS}(r)=v_{ext}(r)+v_H(r)+v_{xc}(r)
\end{equation}
 	\begin{equation}
  \label{eq:equation3}
v_H(r)=\int\frac{n(r)}{|r-r'|}dr'
\end{equation}
 	\begin{equation}
  \label{eq:equation4}
v_{xc}(r)=\frac{\delta E_{xc}}{\delta n(r)}
\end{equation}

Средняя плотность электронов определяется выражением:
 	\begin{equation}
  \label{eq:equation5}
n(r)=\sum_i|\psi_i(r)|^2
\end{equation}
где индекс j пробегает по всех состояниям, которые заполнены в соответствии с принципом Паули.
В случае проведения вычислений магнитных свойств твердых тел необходимо вводить отдельно плотности электронов для случая с s=+1/2 и s=-1/2.
 	\begin{equation}
  \label{eq:equation6}
n^\uparrow(r)=\sum_i|\psi^\uparrow_i(r)|^2, n^\downarrow(r)=\sum_i|\psi^\downarrow_i(r)|^2
\end{equation}

Энергия становится функционалом ${n^\downarrow(r)}$ и ${n^\uparrow(r)}$:
 	\begin{equation}
  \label{eq:equation7}
E_{KS}[n^\uparrow,n^\downarrow]=T_s+E_H+E_{ext}+E_{xc}[n^\uparrow,n^\downarrow]
\end{equation}
	
	Принято использовать вместо ${n^\downarrow(r)}$ и ${n^\uparrow(r)}$ полную электронную плотность и намагниченность:
 	\begin{equation}
  \label{eq:equation8}
n(r)=n^\uparrow(r)+n^\downarrow(r), m(r)=n^\uparrow(r)-n^\downarrow(r)
\end{equation}
	
	Тогда уравнения Кона-Шэма принимают вид:
\begin{equation}
  \label{eq:equation9}
  %\begin{align}
 (-\frac{1}{2}\triangledown^2+V_H+V_{ext}+V_{xc}+B_{xc})\psi_i^\uparrow(r)=\epsilon_i^\uparrow\psi_i^\uparrow(r) \\
(-\frac{1}{2}\triangledown^2+V_H+V_{ext}+V_{xc}+B_{xc})\psi_i^\downarrow(r)=\epsilon_i^\downarrow\psi_i^\downarrow(r)
%\end{align}
 \end{equation}
где ${V_{xc}=\frac{\delta E_{xc}[n,m]}{\delta n(r)}}$ ,${B_{xc}=\frac{\delta E_{xc}[n,m]}{\delta m(r)}}$.
	
	Магнетизм исходит из обменно-корреляционного функционала: одно направление спина оказывается энергетически более выгодно, чем другое.
	Уравнения Кона-Шэма могут рассматриваться как формальное обобщение теории Хартри. Если бы было известно выражение для обменно-корреляционной энергии, было бы возможно точное описание многоэлектронных систем. Оказалось, что можно найти для обменно-корреляционной энергии удачную аппроксимацию. Наиболее простая – аппроксимация локальной плотности (local density approximation - LDA).
 	\begin{equation}
  \label{eq:equation10}
E_{xc}^{LDA}=\int drv_{xc}(n(r))n(r)
\end{equation}
где ${v_{xc}(n(r))}$ - обменно-корреляционная энергия на одну частицу однородного газа.
	
	Существует несколько подходов к усовершенствованию приближения локальной плотности. В одном из них построена теория, которая учитывает неоднородное распределение электронной плотности. Это обобщенно-градиентное разложение (generalized gradient approximation — GGA). Здесь выражение для обменно-корреляционной энергии раскладывается по степеням градиента плотности.
 	\begin{equation}
  \label{eq:equation11}
E_{xc}^{GGA}=\int drf(n(r),|\triangledown n(r)|)n(r)
\end{equation}
где ${f(n(r),|\triangledown n(r)|)}$ - некая функция, для которой получено приближенное выражение. Аналогичные уравнения можно написать для спин-поляризованного обобщенно-градиентного приближения.

	Рассмотрим теперь способ решения уравнений Кона-Шэма. Он является итерационным:

1.	На первом этапе задаются начальные приближения для электронной плотности $n(r)$;

2.	Затем вычисляется потенциал:
 	\[
v_{KS}(r)=v_H+v_{ext}(r)+v_{xc}(r), v_H=\int dr'\frac{n(r)}{|r-r'|}
\]		

3.	На следующем шаге решаются уравнения Кона-Шэма:


\[
 % \begin{align}
 (-\frac{1}{2}\triangledown^2+V_H+V_{ext}+V_{xc}+B_{xc})\psi_i^\uparrow(r)=\epsilon_i^\uparrow\psi_i^\uparrow(r) \\
(-\frac{1}{2}\triangledown^2+V_H+V_{ext}+V_{xc}+B_{xc})\psi_i^\downarrow(r)=\epsilon_i^\downarrow\psi_i^\downarrow(r)
  %\end{align}
 \]



4.	После этого находятся уточненные значения электронной плотности:
\[
n^\uparrow(r)=\sum_i|\psi^\uparrow_i(r)|^2, n^\downarrow(r)=\sum_i|\psi^\downarrow_i(r)|^2
\]
    
    Затем процедура повторяется, пока не будут выполнены критерии сходимости.
    
\section{Практическое применение}
	
	Теперь опишем некоторые практические аспекты решения уравнений Кона-Шема. При численном решении уравнений на собственные числа возникает необходимость в разложении волновых функций в базис. Базисными функциями могут быть как плоские волны, так и наборы локализованных функций (например, гауссовых). Широкое распространение получило использование в качестве базисных функций плоских волн. Их преимущества состоят в том, что они удобны для программирования, ортогональны друг другу и не зависят от позиций атомов. С другой стороны, для корректного представления волновых функций требуется большое число базисных функций (в общем случае — бесконечность). Другой недостаток плоских волн связан с тем, что набор функций в наборе дискретен только в случае периодической системы.
	
	Для волновой функции электрона в периодическом потенциале справедлива теорема Блоха
 	\begin{equation}
  \label{eq:equation12}
\psi_k(r)=e^{ikr}u_k(r)
\end{equation}
где u(k) — периодическая функция, то есть
 	\begin{equation}
  \label{eq:equation13}
u(r)=u(r+R)
\end{equation}
где R — вектор трансляции. В Фурье-разложении такой функции возникают только определенные плоские  волны:
 	\begin{equation}
  \label{eq:equation14}
u_k(r)=\frac{1}{\Omega}\sum_Gc_{k,G}e^{iGr}
\end{equation}
где G — вектор трансляции в обратном пространстве, причем плоские волны, которые появляются в таком разложении, можно представить как сетку в обратном пространстве, причем эта сетка простирается на бесконечность и является дискретной только для периодических систем.  Однако на практике оказывается, что вклады высоких гармоник в разложение пренебрежимо малы. Следовательно, можно ограничить набор теми плоскими волнами, для которых
 	\begin{equation}
  \label{eq:equation15}
\frac{h^2|k+G|^2}{2m_e}\le E_{cut}
\end{equation}
где   — энергия обрезки. Её значение подбирают таким образом, чтобы точность вычислений была приемлемой при как можно меньшем размере базиса.
	
	Уравнения Кона-Шема в базисе плоских волн выглядят следующим образом:
 	\begin{equation}
  \label{eq:equation16}
\sum_GH_{k+G,k+G'}c_{i,k+G}=\epsilon_ic_{i,k+G}
\end{equation}
где матричный элемент Гамильтониана дается формулой:
 	\begin{equation}
  \label{eq:equation17}
\frac{1}{2}|k+G|^2\delta_{G,G'}+V_{ion}(k+G,k+G')+V_H(G-G')+V_{xc}(G+G')
\end{equation}
	
	Рассмотрим слагаемое, связанное с потенциалом ядер. Ядро создает вокруг себя кулоновский потенциал, который выражается простой и известной формулой: 
 	\begin{equation}
  \label{eq:equation18}
V_{nuc}=-\frac{z}{r}
\end{equation}

Однако такая форма потенциала является причиной вычислительных затруднений.  Электроны в атоме можно условно разделить на две группы. К первой группе относятся остовные электроны, локализованные в окрестности ядра и не участвующие в формировании химических связей. Их волновые функции имеют резкие пики вблизи ядра. Ко второй группе принадлежат валентные электроны. Их волновые функции имеют максимумы вдали от ядра и в окрестности ядра осциллируют из-за требования ортогональности.  При образовании связей волновые функции валентных электронов могут значительно перестраиваться. Наличие вблизи ядра острых максимумов волновых функций остовных электронов и осцилляций волновых функций валентных электронов обуславливает появление в Фурье-разложении высоких гармоник, что приводит к большим энергиям отсечки. Решением этой проблемы является введение псевдопотенциала. В методе псевдопотенциала волновые функции остовных электронов заменяются псевдофункциями, гладкими вблизи ядра и совпадающими с «честными» волновыми функциями вдали от него. Таким образом, кулоновский потенциал ядер становится экранированным. Уравнения Кона-Шема теперь решаются только для валентных электронов, которые участвуют в формировании химических связей. Помимо снижения вычислительной затратности за счет учета только валентных электронов, использование псевдопотенциала приводит в уменьшению числа функций базиса за счет исчезновения осцилляций волновых функций в окрестности ядра и, следовательно, уменьшения энергии обрезки.
	
	Все вышесказанное применимо к периодическим системам, таким как объемные кристаллы. Однако на практике исследуемая система часто представляет собой тонкую пленку, квантовую нить, квантовую точку или отдельную молекулу. В этом случае удобно выбирать для трансляции суперячейку, содержащую вакуумные зазоры в направлениях, в которых отсутствует периодичность. Ширина зазоров выбирается таким образом, чтобы соседние структуры не влияли друг на друга. Такой подход обеспечивает дискретный набор базисных волновых функций и позволяет рассчитывать свойства систем пониженной размерности.

Метод функционала плотности получил широкое распространение и в настоящий момент является одним из самых популярных инструментов для расчетов свойств твердых тел и наноразмерных систем.

